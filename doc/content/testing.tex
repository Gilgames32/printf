\chapter{Tesztelés és visszajelzések}

Automatizált teszteknek nem készítettem, a generált képeket szemmel ellenőriztem, és összehasonlítottam a script által generált eredményekkel. Ezen felül visszaváltottam a kép szélességét pixelből milliméterre, megbizonyosodva hogy a program valóban megfelelő szélességű képeket generál. Ezt később a UI-be beépítettem: generálás után az alsó sávban láthatjuk a generált kép méreteit pixelben és milliméterben.

\section{Éles teszt}

Egy ismerősömnek köszönhetően élesben is volt lehetőségem egy plotteren tesztelni a programot. Tesztelés közben merült fel egy már említett probléma, ami a listából kilógó bemeneti képek esetén jelentkezett. Ezt leszámítva a tesztnyomtatás sikeres volt.

\section{Visszajelzések és ötletek}

Kikértem a kör aktívabb tagjainak véleményeit és ötleteit. Pozitív visszajelzéseket kaptam, és érkezett pár ötlet is. Ezekből egy valóban hasznos ötlet volt, hogy lehessen tekercsen kívül fix méretű lapokra is csempézni, hogy irodai nyomtatókkal is működhessen, ha kisebb dokumentumokat akarunk pl. A4-es papírra nyomtatni. Ezt valószínűleg nem lesz kimondottan nehéz implementálni sem.
