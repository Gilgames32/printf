\chapter{Tesztelés és visszajelzések}

Automatizált teszteket nem készítettem, a generált képeket manuálisan ellenőriztem, és összehasonlítottam a script által generált eredményekkel. Ezen felül visszaváltottam a kép szélességét pixelből milliméterre, megbizonyosodva arról, hogy a program valóban megfelelő szélességű képeket generál. Ezt később a UI-be is beépítettem: generálás után az alsó sávban láthatjuk a generált kép méreteit pixelben és milliméterben.

\section{Éles teszt}

Egy ismerősömnek köszönhetően élesben is volt lehetőségem egy plotteren tesztelni a programot. Tesztelés közben merült fel egy már említett probléma is, ami a listából kilógó bemeneti képek esetén jelentkezett. Ez csak a forrásfájlok mennyiségére volt hatással, ezt leszámítva a tesztnyomtatás sikeres volt.

\section{Visszajelzések és ötletek}

Kikértem néhány aktívabb tag véleményét és ötleteit. Pozitív visszajelzéseket kaptam, és érkezett pár ötlet is. Ezekből egy valóban hasznos ötlet volt, hogy lehessen tekercsen kívül fix méretű lapokra is csempézni, hogy irodai nyomtatókkal is működhessen, ha kisebb dokumentumokat akarunk pl. A4-es papírra nyomtatni. Ezt valószínűleg nem lesz nehéz implementálni sem.
