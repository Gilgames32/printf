\pagenumbering{roman}
\setcounter{page}{1}

\selecthungarian
\chapter*{Kivonat}\addcontentsline{toc}{chapter}{Kivonat}

Papírtekercsekre történő nyomtatáskor gyakori eset, hogy a kinyomtatott dokumentumok nem mindig tekercs szélességűek. Ilyenkor egy olyan vágóasztallal vágható elrendezést kell keresni, ahol a keletkező felesleg minél kisebb. Az eredményhez hozzá kell adni térközt és segédvonalakat, melyek mentén a kinyomtatott papírt fel lehet vágni. A nyomtatók hivatalos szoftverei sokszor lassúak és nem nyújtanak elegendő lehetőséget a precíz konfigurációra, így felmerül az igény egy nyomtatást optimalizáló szoftver kifejlesztésére.

\vfill
\selectenglish
\chapter*{Abstract}\addcontentsline{toc}{chapter}{Abstract}

Often times when printing on paper rolls, the source documents do not match the full width of the roll. In such cases a placement with minimal waste has to be found, which can be cut by only guillotine cuts. The result must feature padding between images and guides, around which the document can be cut to the desired size. Official softwares of plotters are often slow and lack features for precise configurations, creating a need for the development of print optimization software.

\vfill
\selectthesislanguage

\newcounter{romanPage}
\setcounter{romanPage}{\value{page}}
\stepcounter{romanPage}