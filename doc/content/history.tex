% Előzmények (irodalomkutatás, hasonló alkotások), az ezekből levonható következtetések

% előző scriptek
% hivatalos programok is csak griden
% hol használnak még ilyet nyomtatáson kívül (üveg)
% ládapakolás, np nehéz
% extra megkötés, vágóasztallal vágás
% talált irodalom
% felhasznált pszeudokód? idk
% egyéb algoritmusok a strip packingen kívül

\chapter{Előzmények}

\section{Hivatalos szoftverek}

A Canon imagePROGRAF PRO-4600 plotterhez több különféle nyomtatási segédprogramot is kaptunk. Megpróbáltunk áttérni a használatukra, viszont alapvető funckiók hiányoztak belőlük. A legtöbb program csak kézzel való elrendezést tett lehetővé, a mi esetünkben ez viszont egy visszalépést jelentett volna a szkripttel szemben.

\section{A Photoshop szkriptről}

TODO normális formázás és vesszők

A szkript által használt algoritmus felettébb egyszerű. A dokumentumot egy rács mentén cellákra osztjuk fel, ahol minden cella mérete a kinyomtatni kívánt kép méretének felel meg.
Nem mindegy, hogy a cellák fekvő vagy álló tájolással vannak elhelyezve a dokumentumon, ugyanis ez különböző mennyiségű felesleget eredményezhet.

TODO kép

Az orientáció kiszámításához megvizsgáljuk mindkét esetet, egy teljesen kitöltött sorra. Legyen a tekercs szélessége $W$, a keletkező felesleg pedig $A$. Vegyünk egy képet, ahol a kép szélessége $w$, magassága $h$, orientációja $r$. Ekkor a keletkező felesleg

$$m_1 = \bigg\lfloor \cfrac{W}{w} \bigg\rfloor, \qquad A_1 = (W - m_1 * w) * h,$$

\noindent illetve ellentétes orientáció esetén

$$m_2 = \bigg\lfloor \cfrac{W}{h} \bigg\rfloor, \qquad A_2 = (W - m_2 * h) * w.$$


A kevesebb felesleget eredényező eset határozza meg az orientációt. Az orientáció alapján kiszámolható az rács oszlopainak száma, $m$. Ez $n$ kinyomtatni kívánt kép esetén $k$ oszlopot eredményez, ahol

$$k = \bigg\lceil \cfrac{n}{m} \bigg\rceil.$$

\noindent Amennyiben

$$ n \mod m \neq 0,$$

\noindent úgy ez az eljárás üres cellákhoz, azaz papírfelesleghez fog vezetni. 

TODO kép

TODO fix indentation

Emiatt született meg az opció a mennyiség korrigálására, mely megnöveli $n$-t úgy, hogy a korrigálás előtti esetleges üres cellákba is kerüljenek képek:

$$n' = k * m.$$

TODO kép

Ez a megoldás sajnos nem tökéletes. Ha az utolsó sorunk cellái nincsenek teljesen kitöltve, akkor nem biztos, hogy $r$ orientáció valóban kevesebb felesleggel fog járni.

TODO még magyarázat hogy ezt hogyan vagy hogy erre még visszatérek idk

\section{Különböző méretű képek elrendezése}

A különböző méretű dobozok fix területre történő elhelyezése a ládapakolási problémára vezethető vissza. A mi problémánk egy speciális esete ennek. A tekercs szélessége fix, a magassága viszont tetszőleges, de minimalizálni szeretnénk. Az eredménynek vágóasztallal vághatónak kell lennie.

\subsection{Guillotine vágás}

A guillotine vágás egy olyan folyamat, ahol egy nagyobb lemezt vágunk fel előre megadott, kisebb téglalapokra, guillotine vágásokkal. A guillotine vágások, más néven széltől-szélig vágások egyik oldaltól a másik oldalig tartó egyenes vágások, melyek az téglalapot két részre osztják fel. Ezt az eljárást elsősorban üveg, acél, fa illetve kartonpapír lapok felvágásakor alkalmazzák. Több különböző szempont alapján is próbálhatjuk optimalizálni a problémát, például maximalizálni a kapott elemek területét, értékét vagy számát illetve minimalizálni a szükséges vágások számát vagy a keletkező felseleget.

A guillotine vágás esetén a lemezek mérete előre meg van adva, és a kisebb téglalapok mennyisége nincsen megkötve. A mi esetünk ugyan nagyban hasonlít a guillotine vágásra, leszámítva, hogy a képek mennyisége fix, és csak a tekercs szélessége adott. Ennek tudatában folytattuk tovább a kutatómunkát barátommal és kollégámmal, Orbán Levente Lászlóval.

% TODO forrás megjelölés: https://en.wikipedia.org/wiki/Guillotine_cutting

\subsection{Tekercsre pakolás probléma}

A tekercsre pakolás probléma egy két dimenziós geometriai minimalizálási probléma. Adott egy tekercs fix szélességgel és tetszőleges magassággal, valamint a tekercsre helyezendő téglalapok. Meghatározandó egy elrendezés, mely átfedés nélkül helyezi el a téglalapokat a tekercsre, minimalizálva a felhasznált tekercsrész magasságát. Legtöbbet a textil- és papírtekercsek feldolgozásával foglalkozó kontextusokban találkozhatunk ilyen megoldásokkal.

Sajnos a tekercsre pakolási probléma erősen NP-nehéz, mert magában tartalmazza a ládapakolási problémát fix magasságú elemekkel. 

% TODO forrás megjelölés: https://en.wikipedia.org/wiki/Strip_packing_problem

A feladatunk tehát visszavezethető a tekercsre pakolás és a guillotine vágás keveréke, egy tekercsre kell guillotine-nal vágható elrendezésben pakolnunk az elemeinket. Szerencsénkre ez már egy ismert speciális esete a tekercsre pakolásnak, és több különféle algoritmus is létezik már rá.

\section{A választott algoritmusok}

Egyetlen méret nyomtatása esetén a régi szkript kapcsán ismertetett rácsra helyezés javított változata már egy optimális megoldás, és megvan az opciónk, hogy korrigáljuk a mennyiséget az esetleges ki nem töltött cellák alapján. 
Különböző méretű képek esetén a probléma erősen NP-nehéz voltából adódóan úgy döntöttünk, hogy nem fogunk optimális megoldást keresni, ugyanis fontos szempont volt a program futásának ideje, és mert az esetek túlnyomó többségében nem nyomtatunk több különböző méretben egyszerre.  

Leventének köszönhetően találtuk meg a 

% TODO https://sci-hub.se/10.1016/j.ipl.2015.08.008

% TODO egyéb heurisztikák