% Előzmények (irodalomkutatás, hasonló alkotások), az ezekből levonható következtetések

% előző scriptek
% hivatalos programok is csak griden
% hol használnak még ilyet nyomtatáson kívül (üveg)
% ládapakolás, np nehéz
% extra megkötés, vágóasztallal vágás
% talált irodalom
% felhasznált pszeudokód? idk
% egyéb algoritmusok a strip packingen kívül

\chapter{Előzmények}

\section{Hivatalos szoftverek}

A Canon imagePROGRAF PRO-4600 plotterhez több különféle nyomtatási segédprogramot is kaptunk. Megpróbáltunk áttérni a használatukra, viszont alapvető funckiók hiányoztak belőlük. A legtöbb program csak kézzel való elrendezést tett lehetővé, a mi esetünkben ez viszont egy visszalépést jelentett volna a szkripttel szemben.

\section{A Photoshop szkriptről}

TODO normális formázás és vesszők

A szkript által használt algoritmus felettébb egyszerű. A dokumentumot egy rács mentén cellákra osztjuk fel, ahol minden cella mérete a kinyomtatni kívánt kép méretének felel meg.
Nem mindegy, hogy a cellák fekvő vagy álló tájolással vannak elhelyezve a dokumentumon, ugyanis ez különböző mennyiségű felesleget eredményezhet.

TODO kép

Az orientáció kiszámításához megvizsgáljuk mindkét esetet, egy teljesen kitöltött sorra. Legyen a tekercs szélessége $W$, a keletkező felesleg pedig $A$. Vegyünk egy képet, ahol a kép szélessége $w$, magassága $h$, orientációja $r$. Ekkor a keletkező felesleg

$$m_1 = \bigg\lfloor \cfrac{W}{w} \bigg\rfloor, \qquad A_1 = (W - m_1 * w) * h,$$

illetve ellentétes orientáció esetén

$$m_2 = \bigg\lfloor \cfrac{W}{h} \bigg\rfloor, \qquad A_2 = (W - m_2 * h) * w,$$


A kevesebb felesleget eredényező eset határozza meg az orientációt. Az orientáció alapján kiszámolható az rács oszlopainak száma, $m$. Ez $n$ kinyomtatni kívánt kép esetén $k$ oszlopot eredményez, ahol

$$k = \bigg\lceil \cfrac{n}{m} \bigg\rceil.$$

Amennyiben

$$ n \mod m \neq 0,$$

úgy ez az eljárás üres cellákhoz, azaz papírfelesleghez fog vezetni. 

TODO kép

Emiatt született meg az opció a mennyiség korrigálására, mely megnöveli $n$-t úgy, hogy a korrigálás előtti esetleges üres cellákba is kerüljenek képek.

$$n' = k * m$$

TODO kép

Ez a megoldás sajnos nem tökéletes. Ha az utolsó sorunk cellái nincsenek teljesen kitöltve, akkor nem biztos, hogy $r$ orientáció valóban kevesebb felesleggel fog járni.

TODO még magyarázat hogy ezt hogyan vagy hogy erre még visszatérek idk

\section{Különböző méretű képek elrendezése}

A különböző méretű dobozok fix területre történő elhelyezése a ládapakolási problémára vezethető vissza. A mi problémánk egy speciális esete ennek. A tekercs szélessége fix, a magassága viszont tetszőleges, de minimalizálni szeretnénk. Az eredménynek vágóasztallal vághatónak kell lennie.

\subsection{Guillotine vágás}

\subsection{Szalagra csempéző algoritmus}

\subsection{Egyéb algoritmusok}