\chapter{\bevezetes}

\section{Problémafelvetés}

A Schönherz kollégiumban hetente megannyi plakát kerül kinyomtatásra. Ezt a feladatot a \vik lelkes hallgatói látják el, kisebb irodai nyomtatók mellett egy Canon imagePROGRAF PRO-4600 plotteren is. A plotteren különböző célokra különböző szélességű és anyagtulajdonságú papírtekercsekre nyomtatunk. Leggyakrabban egy adott, kisebb méretű dokumentumot kell nagy mennyiségben kinyomtatnunk. Ha ezeket próbálnánk meg egyesével kinyomtatni, a nyomtató egymás fölé helyezve nyomtatná ki a dokumentumokat, melyek méretei eltörpülnek a tekecs teljes szélessége mellett, így ez hatalmas papírfelesleget eredményezne.

Ilyen esetekre szolgált egy \href{fig:old_ui}{Adobe Photoshop script}, mely négyzetrácsszerűen felosztotta a papírtekercset, és a rács celláiba helyezte el a dokumentumokat. A szkript emellet egyéb hasznos funkciókkal is rendelkezett, például segédvonalakkal látta el az elkészült dokumentumot a cellák mentén, ami lehetővé tette a precíz vágást a nyomdai vágóasztalokon.
Ez a szkript több hallgató keze műve volt, hat különböző verzióját egyszerre használtuk, mert egyik sem volt teljesen megfelelő minden felhasználási esetre. Csatlakozásomkor ezt a szkriptet saját magam is refaktoráltam, kijavítva az előző verziók hibáit, egyesítve funkcióikat illetve ellátva néhány új extra funkcióval. A projekt nyílt forráskódú és a \href{https://github.com/SCH-KB-PR/kbpr-ps}{https://github.com/SCH-KB-PR/kbpr-ps} oldalon érhető el.

Felmerült az is, hogy keressünk már létező, erre specializálódott szoftvert. A régi plotter cseréje után erre sort is kerítettünk, az új mellé különféle hivatalos szoftvert kaptunk, melyek hasonló célokra voltak tervezve. Sajnos ezek a szoftverek több sebből is véreztek, rengeteg számunkra fontos funkció hiányzott belőlük, a szkripthez hasonlóan egyik sem volt minden szempontból működőképes, nem beszélve arról, hogy nem volt hozzáférésünk a forráskódhoz, ellehetetlenítve ezzel a sajátos felhasználásunkhoz elengedhetetlen extra funkciókat, és kizárólag Windows operációs rendszerek alatt futottak.

A refaktorált szkript már egy nagy erőlelépést jelentett, így jobb híján maradtunk a használatánál. Tudtuk, hogy nem tökéletes, főkét a Photoshop hatalmas overheadje miatt a dokumentumok generálása túl sokáig tartott, és volt, hogy a rendelkezésre álló 8 GB memória sem volt elég egy teljes dokumentumhoz. Sajnos az Adobe ExtendScript szkriptnyelv jellegéből adódóan felettébb limitáltak a lehetőségek bármi nemű optimalizálásra. Gyakorta előjöttek olyan esetek is, hogy különböző méretű dokumentumokat szerettünk volna nyomtatni, de a szkript csak egységes cellaméretekkel volt képes dolgozni, külön nyomtatni pedig némi extra papírfelesleggel járt.

\begin{figure}[!h]
    \includegraphics[width=\textwidth]{figures/script_showcase.png}
    \label{fig:old_ui}
    \caption{A refaktorált Photoshop script felhasználói felülete}
\end{figure}

\section{Célkitűzés}

Az volt a célom, hogy a nyomtatást egyszerűbbé, gyorsabbá és hatékonyabbá tegyem, törekedve a minimális keletkező papírfeleslegre, a szkriptek funkcióit megtartva illetve egyéb hasznos ötletekkel kibővítve. 

\section{A feladat értelmezése}

A feladat egy dokumentumok nyomtatását optimalizáló szoftver elkészítése, ami minimalizálja az elrendezésből adódó papírfelesleget, törekedve a program lefutásának gyorsaságára és annak könnyen használhatóságára.

\section{A feladat indokoltásga}

A plotterekhez kapott hivatalos szoftvereken kívül nem találtunk más, a célra megfelelő eszközt az interneten. Ebből adódóan úgy láttam, hogy indokolt lenne egy erre specializálódó programot készíteni, és hogy a programomat egy nyílt forráskódú projektként szeretném karbantartani, hogy más nyomdák is át tudjanak térni a használatára.

\section{A dolgozat felépítése}

TODO
%A továbbiakban szeretnék részletesebben kitérni a program specifikációjára, a rendszer tervére, annak megvalósítására és a felmerült nehézségekre, majd a szoftver tesztelésre, értékelésére, végül a projekt jövőjére. Mellékletként egy rövid felhasználói kézikönyvet is készítettem.
