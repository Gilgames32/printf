\chapter{Felhasználói kézikönyv}

\section{A program kezelése}

A felhasználói felület intuitív, szinte minden el van látva címkékkel. 
A képek mérete melletti pipával tudjuk állítani, hogy a méret állításakor tartsa-e az eredeti méretarányt.
A legördülő menükben tudunk az előbeállítások közül választani. 
Az előnézetet a görgővel tudjuk nagyítani és az egérrel lehet mozgazni.

\section{Az előbeállítások}

Az előbállítások a \texttt{presets} mappában találhatók, almappákra bontva. Minden előbeállítás JSON formátumban értendő. A mappák és a bennük szerplő JSON fájlok kulcsai az alábbiak:
\begin{itemize}
    \item \texttt{document}
    \begin{itemize}
        \item \texttt{name}: az előbeállítás neve
        \item \texttt{roll\_width\_mm}: a tekercs szélessége milliméterben
        \item \texttt{resolution\_ppi}: a dokumentum felbontása pixel per inchben
        \item \texttt{margin\_mm}: a margó milliméterben, nem lesz része a dokumentumnak
        \item \texttt{gutter\_mm}: a képek közti térköz milliméterben
        \item \texttt{correct\_quantity}: ha \texttt{true}, a rács maradék cellái ki lesznek töltve képekkel\footnote{amennyiben az eredeti algoritmust használjuk, ami rácsba rendezi a bemeneti képeket; az ilyenkor keletkező maradék helyet hivatott kitölteni}
        \item \texttt{guide}: ha \texttt{true} és \texttt{gutter\_mm} \begin{math}
            \geq 0
        \end{math}, akkor segédvonalakat rak a képek sarkaiba
    \end{itemize}

    \item \texttt{mask}
    \begin{itemize}
        \item \texttt{name}: az előbeállítás neve
        \item \texttt{path}: a maszk kép fájl elérési útvonala
    \end{itemize}

    \item \texttt{image}
    \begin{itemize}
        \item \texttt{name}: az előbeállítás neve
        \item \texttt{amount}: a nyomtatni kívánt mennyiség
        \item \texttt{width}: a szélesség milliméterben
        \item \texttt{height}: a magasság milliméterben
        \item továbbá beágyazhatunk más kategóriákra vonatkozó beállításokat is
    \end{itemize}
\end{itemize}