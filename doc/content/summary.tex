\chapter{Összegzés}

A programra pozitív visszajelzések érkeztek minden irányból. Ugyan még nem 100\%-ban teljes, de már használható, és a közeljövőben szeretném beterjeszteni a használatát a rendeléseinkhez.

\section{Kitekintés}

Az érkezett ötleteken kívül saját terveim is vannak a jövőre nézve.

A legambíciózusabb ezek közül, hogy tudjunk a programon keresztül nyomtatni, ne kelljen feltétlenül exportálni. Erre a \href{https://en.wikipedia.org/wiki/CUPS}{CUPS} könyvtár segítségével lenne mód.

Egy másik érdekesebb feladat lenne megoldani, hogy PDF fájlokat is használhassunk mint bemenet. Ennek a nehézsége, hogy nem tárolhatjuk egy pixelmátrixban, mivel többségében vektorgrafikus. Szerencsére ez nagyon jól működne a jelenlegi cache és filter kezeléssel. További komplikációkat okozhat viszont a PDF jellegéből adódó sajátosságok, pl. több oldal kezelése.

Ezek mellett pár apróbb funkcionális és használatot elősegítő funkciót szeretnék hozzáadni, pl. jobban nagyítható előnézet, extra konfigurációk a segédvonalaknak és egyéb alapértelmezett értékeknek. Továbba szeretnénk mihamarabb implementálni a strip tiling algoritmust is.

Amennyiben lehetőségem van rá, szeretnék felkeresni más nyomdákat is, és kikérni a véleményüket, konzultálni velük, hogy milyen funkciókat hiányolnak vagy milyen funkciókat használnak, amit tudnék implementálni.

\section{A forráskódról}

A projekt nyílt forráskódú és a GNU General Public License v3.0 alatt érhető el a \href{https://github.com/Gilgames32/printf}{https://github.com/Gilgames32/printf} címen. Szívesen fogadok minden hozzájárulást a projekthez. 

Ugyan nem szeretném, de az egyetemi keretek és az önálló laboratórium miatt elkülöníteni kettőnk munkáját egymástól. A projektet 2025 február közepe óta egyedül csinálom, a build rendszeren és egy-két közösen hozott modellbeli döntésen kívül a projektben mindent én csináltam, ez GitHubon a committok között is jól látszik.

\section{Köszönet}

Köszönet barátomnak, Orbán Levente Lászlónak, hiszen nélküle ez a projekt most nem létezne, és nem hiszem, hogy egy hasonlóan hasznos és érdekes önálló laboratórium témát tudtam volna választani.
