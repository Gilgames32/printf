\chapter{Specifikáció}

A feladatom tehát egy dokumentumok nyomtatását optimalizáló szoftver volt, ami minimalizálja az elrendezésből adódó papírfelesleget, ügyelve a program teljesítményére és annak könnyen használhatóságára.

\section{Funkcionális követelmények}
\begin{itemize}
    \item A rendszer legyen képes egy tekercs szélességű dokumentum elkészítésében.
    \item A dokumentumon a kívánt mennyiségben szerepeljen(ek) a megadott kép(ek).
    \item A képek a kívánt fizikai méretben szerepeljenek.
    \item A program legyen képes különböző pixelsűrűségeket (mn. dpi/ppi) kezelni.
    \item A képek fizikai mérete lehessen előre megadott szabvány méretű (pl. A4).
    \item A tekercsek fizikai mérete lehessen előre megadott szélességű.
    \item A képek közé lehessen állítható méretű térközt rakni.
    \item A dokumentum széleire lehessen állítható méretű margót tenni.
    \item A dokumentum lehessen vágóasztallal (széltől szélig) vágható.
    \item Legyen a képek vágását segítő segédvonal, ami mentén a dokumentum vágóasztallal vágható.
    \item A dokumentum lehessen nyomtatható és/vagy lehessen képi formátumban exportálni.
\end{itemize}

\section{Nem funkcionális követelmények}
\begin{itemize}
    \item A program törekedjem a gyors dokumentumgenerálásra. 
    \item A programhoz tartozzon egy grafikus felhasználói felület.
    \item A felhasználói felület legyen intuitív.
    \item A program törekedjen az effektív memóriahasználatra.
    \item A program működjön Linux operációs rendszer alatt (is).
    \item A program legyen befogadóképes a jövőbeli új funkciókra.
\end{itemize}