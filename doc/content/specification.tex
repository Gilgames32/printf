\chapter{Specifikáció}

A feladat egy olyan szoftver készítése, mely képes adott szélességű és tetszőleges magasságú tekercsre a tekercsnél kisebb dokumentumokat elhelyezni úgy, hogy minél kevesebb papírfelesleg keletkezzen, azaz az eredmény magassága minél kisebb legyen. 

\section{Funkcionális követelmények}
\begin{itemize}
    \item A rendszer legyen képes egy tekercs szélességű dokumentum elkészítésére.
    \item A dokumentumon a kívánt mennyiségben szerepeljen(ek) a megadott kép(ek).
    \item A képek a kívánt fizikai méretben szerepeljenek.
    \item A program legyen képes különböző pixelsűrűségű dokumentumokat (mn. dpi/ppi) kezelni és készíteni.
    \item A képek fizikai mérete lehessen előre megadott szabvány méretű (pl. A4).
    \item A tekercsek fizikai mérete lehessen előre megadott szélességű.
    \item A képek közé lehessen állítható méretű térközt rakni.
    \item A dokumentum széleire lehessen állítható méretű margót tenni.
    \item A dokumentumnak lehessen minimum, illetve maximum magasságokat megadni. 
    \item A dokumentum legyen vágóasztallal vágható. Ez azt jelenti, hogy a dokumentumot legalább egy helyen el kell tudni vágni vágóasztallal, azaz széltől szélig, egyetlen egyenes vonallal. Az így keletkező tovább osztható szeleteknek is vágóasztallal vághatónak kell lennie.
    \item A dokumentum tartalmazzon képek vágását elősegítő segédvonalakat, melyek mentén a dokumentumot vágóasztallal fel lehet vágni.
    \item A segédvonalak vastagsága legyen konfigurálható.
    \item A segédvonalak hossza legyen konfigurálható, azaz be lehessen állítani, hogy a képek sarkától számítva a segédvonalak meddig futnak be a kép mentén.
    \item A dokumentum lehessen nyomtatható és/vagy lehessen képként exportálni.
\end{itemize}

\section{Nem funkcionális követelmények}
\begin{itemize}
    \item A program törekedjen a gyors dokumentumgenerálásra. 
    \item A programhoz tartozzon egy grafikus felhasználói felület.
    \item A grafikus felület nyújtson kényelmes előnézetet a generált képről, mielőtt az exportálásra kerülne. 
    \item A felhasználói felület legyen intuitív.
    \item A program törekedjen az effektív memóriahasználatra.
    \item A program legyen képes elterjedt nyomtatási formátumú képekkel dolgozni (PNG, JPG, PDF). 
    \item A program működjön Linux operációs rendszer alatt (is).
    \item A program legyen befogadóképes a jövőbeli új funkciókra.
    \item A program jelezze, ha generálás közben hibába ütközne, és kezelje azt a program leállása nélkül.
    \item A program tartalmazzon elterjedt szabvány méreteket és gyakran használt előbeállítást.
\end{itemize}